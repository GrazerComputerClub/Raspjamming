\subsection{Blockly-gPIo - Visuelle Programmierumgebung}

Blockly-gPIo ist eine visuelle Programmierumgebung die �ber eine Webseite, also den lokalen Browser bedient wird. %Es basiert auf dem Projekt Blockly von Google.  
�hnlich wie bei Scratch oder PocketCode, soll es einen sehr leichten Einstieg in die Programmierung erm�glichen.\\ 
Durch einfaches hinzuf�gen von fertigen Funktionsbl�cken und verbinden der Bl�cke sowie einfach Parametrierung k�nnen Programmieraufgaben gel�st werden.\\

Technologisch besteht Blockly-gPIo aus einer Web-Seite, die beim Raspjamming Image auf einem Webserver am System verf�gbar ist \url{http://raspberrypi.local/Blockly-gPIo}. Der Browser erzeugt aus dem grafischen Programmierung ein Python-Programmcode.\\
Am Raspberry Pi gibt es weiters einen Server-Dienst, der f�r die Ausf�hrung des Python-Programms zust�ndig ist. Dieser ist ebenfalls am Raspjamming Image vorinstalliert und entsprechend konfiguriert.\\
F�r den Zugriff auf die Raspberry Pi GPIOs und angeschlossene Sensoren gibt es fertige Funktionsbl�cke. Ein besonderes Merkmal ist, dass das Programm auch in einem Simulationsmodus gestartet werden kann. Dann werden die GPIO-Zust�nde grafisch im Browser dargestellt.

   