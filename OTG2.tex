\subsection{Client - Statische IP-Adresse und DHCP Server}

Die folgende Einstellungen m�ssen nicht gemacht werden, erleichtern aber das Arbeiten. Der Raspberry Pi Zero hat dann eine statische IP-Adresse und kann leichter angesprochen und auch die Internetverbindung freigegeben werden (vor allem mit Microsoft Windows 7 und 10).\\
Am Client-System, also der Raspberry Pi Zero, kann die Netzwerkadresse, der Gateway und ein DNS-Server eingestellt werden. Dieser Schritt ist unbedingt n�tig wenn die Internetverbindung unter Windows dem Ger�t zur Verf�gung gestellt werden soll. Die IP-Adresse die eingestellt wird, muss f�r Windows 7 aus dem Bereich 192.168.137.* sein (z.~B. 192.168.137.10). Der Gateway ist die IP-Adresse des Host-PC. Als DNS-Server kann z.~B. der Server von Google mit der Adresse 8.8.8.8 verwendet werden. Die Einstellungen k�nnen in der Konfigurationsdatei f�r den DHCP-Client definiert werden.\\ 
Die IP-Adresse des Host-Computers kann via eines DHCP-Server konfiguriert werden. Das hat den Vorteil, dass dort keine Konfiguration des Netzwerks erfolgen muss. Es wird zur weiteren Einrichtung ein Terminalzugang zum Einplatinencomputer ben�tigt und eine Internetverbindung sollte bestehen damit man den DHCP-Server installieren kann.\\
Per SSH-Client kann eine Verbindungen zum Raspberry Pi mit dem Befehl "'ssh pi@raspberrypi.local"' hergestellt werden. %Nun kann die Konfiguration abgeschlossen werden.   

\begin{console}
	sudo vi /etc/dhcpcd.conf
\end{console}

Folgende Zeilen m�ssen am Ende der Datei eingef�gt werden:\\

\filename{/etc/dhcpcd.conf [-rw-r-{-}r-{-} root root]}
\begin{file}
	
# define static profile for Pi 
profile static_usb0
static ip_address=192.168.137.10/24
static routers=192.168.137.1
static domain_name_servers=8.8.8.8

# static profile on usb0
interface usb0 
fallback static_usb0
\end{file}


\begin{console}
	sudo apt-get update
	sudo apt-get install isc-dhcp-server
\end{console}

%Falls die Internetverbindung am Raspberry Pi Zero nicht m�glich ist, kann der Server auch manuell heruntergeladen, auf der Boot-Partition gespeichert und installiert werden.\\
%Install-Datei:
%\url{http://archive.raspbian.org/raspbian/pool/main/i/isc-dhcp/isc-dhcp-server_4.3.5-3_armhf.deb}
%	
%	\begin{console}
%		sudo dpkg -i /boot/isc-dhcp-server_4.3.1-6+deb8u2_armhf.deb
%	\end{console}
%	
%	Nach der Installation kann der DHCP-Server parametriert werden. 
	
\begin{console}
	sudo vi /etc/dhcp/dhcpd.conf
\end{console}

%ddns-update-style none;

%#option domain-name "example.org";
%#option domain-name-servers ns1.example.org, ns2.example.org;

%default-lease-time 600;
%max-lease-time 7200;
%
%log-facility local7;


Folgende Zeilen m�ssen am Ende der Datei eingef�gt werden:\\

\filename{/etc/dhcp/dhcpd.conf [-rw-r-{-}r-{-} root root]}
\begin{file}
subnet 192.168.137.0 netmask 255.255.255.0{
	range 192.168.137.2 192.168.137.9;
	option broadcast-address 192.168.137.255;
}

#USB OTG Host-PC set IP-Address to 192.168.137.1
host USB_OTG_HOST {
	hardware ethernet 00:01:02:03:04:05;
	fixed-address 192.168.137.1;
}
\end{file}

\begin{console}
	sudo reboot
\end{console}
