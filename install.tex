Folgende Programme sind auf dem vorbereiteten Image bereits installiert. Demnach m�ssen die angef�hrten Anweisungen nur bei einem Raspbian Image ausgef�hrt werden. Auf die gr��eren Programme \textit{CodeLite} und \textit{Code::Blocks} wurde allerdings beim vorinstallierten Image verzichtet, da bei der Anleitung Geany verwendet wird. 

%\textbf{Raspberry Pi:}
\begin{console}
sudo apt-get update
sudo apt-get -y upgrade
sudo apt-get install geany vim wiringpi pigpio git build-essential automake
sudo apt-get install minicom screen
sudo apt-get install python-dev python-openssl rpi.gpio
sudo apt-get install python3-dev python3-openssl python3-rpi.gpio python3-gpiozero python3-pip
sudo apt-get install mono-complete
sudo apt-get install codelite codeblocks
sudo apt-get clean
sudo reboot
\end{console}


Die Abk�rzung \textit{IDE} bezeichnet eine integrierte Entwicklungsumgebung (Integrated Development Environment). Sie stellt umfassende Funktionen und Programme zur Entwicklung von Programmen bereit. Die Hauptkomponenten sind Editor, Compiler und Debugger.\\  
Eine Versionsverwaltung ist ein System, das zur Verwaltung, Archivierung und Erfassung von �nderungen an Source-Dateien verwendet wird. GitHub ist ein webbasierter Online-Dienst zur Versionsverwaltung, der viele Software-Entwicklungsprojekte bereitstellt.\\
\textit{CodeLite} ist eine freie plattform�bergreifende IDE, die auf die Programmiersprachen C, C++, PHP und JavaScript (Node.js) spezialisiert ist. \textit{Code::Blocks} ist eine freie plattform�bergreifende IDE f�r die Programmiersprachen C, C++ und Fortran.\\
\textit{Geany} ist ein Texteditor mit grundlegenden Funktionen. Es wurde entwickelt, um eine kleine und schnelle IDE bereitzustellen. Nachteil gegen�ber einer IDE wie \textit{CodeLite} ist vor allem das Fehlen eines Debuggers und Probleme bei multiplen Projektdateien. Der Benutzer muss sich darum k�mmern, dass die Kompileranweisung alle Sourcecode Komponenten einschlie�t. Kompilierfehler werden im Source nicht hervorgehoben, sondern nur in einem Fester ausgegeben. F�r die einfachen Beispiele in dieser Anleitung ist es aber gut geeignet. Das Programm unterst�tzt alle wichtigen Entwicklungsumgebungen wie C, C++, C\# und Python.\\
\textit{WiringPi} und \textit{pigpio} sind zwei C-Bibliotheken, die das Arbeiten mit den GPIOs der Raspberry Pi erm�glichen. Sie sind nicht kompatibel, man muss sich also f�r eine entscheiden. In der Anleitung wird ausschlie�lich die \textit{WiringPi} Bibliothek benutzt. \textit{git} ist ein Versionsverwaltungsystem das Zugriff auf Source-Dateien von Projekte erm�glicht. Diese werden zumeist auf Online-Dienst GitHub zur Verf�gung gestellt. Das Paket \textit{build-essential} enth�lt GNU C und C++ Compiler sowie die GNU-C-Bibliothek um C/C++-Projekte kompilieren bzw. erstellen zu k�nnen.\\ 
\textit{minicom} und \textit{screen} sind Programme, um auf der seriellen Schnittstelle (UART) der Raspberry Pi kommunizieren zu k�nnen.\\
\textit{python-dev} bzw. \textit{python3-dev} enth�lt Bibliothek und Entwicklungswerkzeuge zum Erstellen von Python-Scripte, sowie den Python-Interpreter selbst. In der Anleitung wird ausschlie�lich die aktuelle Version 3 von Python verwendet. \textit{rpi.gpio} ist eine Python Bibliothek die Basisfunktionalit�ten der GPIOs unterst�tzt. \textit{gpiozero} ist eine aktuelle Python Bibliothek die viele Funktionen der GPIOs zur Verf�gung stellt und von der Raspberry Pi Foundation empfohlen wird. Bei den Beispielen wird ausschlie�lich diese Library verwendet.\\
\textit{mono-complete} enth�lt Compiler, Librarys und die Runtimeumgebung um CIL (Common Intermediate Language) Bytecode, auch als Assemblies bekannt, erzeugen und ausf�hren zu k�nnen. Es wird f�r die Erstellung von C\# Programmen ben�tigt.
