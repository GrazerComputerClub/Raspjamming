
Als erstes kann das Raspberry Pi Software Configuration Tool (raspi-config)  gestartet werden. Dazu verwendet man den Aufruf "`sudo raspi-config"'.\\
Der Men�punkt "`Change User Password"' �ndert das Passwort f�r den Benutzer "`pi"'.\\
Danach sollte man die Regionseinstellungen mit dem Men�punkt "`Localisation Options"' einstellen. Im folgenden Untermen� kann man mit "`Change Locale"' die Sprache und Zeichensatz des Systems setzen. Hier w�hlt man z.~B. "`de\_AT.UTF-8 UTF8"' f�r �sterreich oder "`de\_DE.UTF-8 UTF8"' f�r Deutschland. "`en\_GB.UTF-8 UTF8"' und  "`C.UTF-8"' sind bereits ausgew�hlt und k�nnen zus�tzlich aktiv sein. Im n�chsten Fenster kann man dann die Sprache "`de\_AT.UTF-8 UTF8"' bzw. "`de\_DE.UTF-8 UTF8"' als Standardeinstellung �bernehmen.\\ 
Nun kann man mit "`Change Timezone"' die aktuelle Zeitzone ausw�hlen. F�r das geografische Gebiet kann man "`Europe"' ausw�hlen. Danach kann man als Zeitzone die Stadt "`Vienna"' oder "`Berlin"' ausw�hlen.\\
Bei der dritten Einstellung "`Change Keyboard Layout"' kann man das Tastatur-Layout aktualisieren.
Die vierte Einstellung "`Change Wi-fi Country"' passt die WLAN-Einstellungen an die L�ndervorgaben an (verf�gbare Kan�le und Frequenzen). Hier kann "`AT Austria"' f�r �sterreich gew�hlt werden.\\ 
Nun kann noch die Speicherzuweisung der GPU auf lediglich 32~MB gesetzt werden. Dazu w�hlt man den Men�punkt  "`Advanced Options"' und "`Memory Split"', dann kann man 32 f�r die Speichergr��e eingeben.\\ 
