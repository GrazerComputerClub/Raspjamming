
%Alternative: https://github.com/RPi-Distro/pi-gen 	

%\textbf{Raspberry Pi:}
\begin{console}
sudo -i
apt-get update && apt-get -y upgrade
apt-get install mono-complete
apt-get install python-dev python-openssl rpi.gpio
apt-get install python3-dev python3-openssl python3-rpi.gpio python3-gpiozero python3-pip
apt-get install raspi-gpio geany rxvt-unicode vim wiringpi pigpio git build-essential automake cmake
apt-get install minicom screen avrdude esptool
apt-get install fbi
lighttpd
%apt-get install zsh
%wget -O .zshrc https://git.grml.org/f/grml-etc-core/etc/zsh/zshrc
%chmod --shell /bin/zsh pi
apt-get clean
cd ~
mkdir Projekte
cd Projekte
git clone https://github.com/mstroh76/TM1637Display
git clone https://github.com/mstroh76/Sensors-WiringPi.git
wget --trust-server-names https://goo.gl/isrNeJ
git clone https://github.com/chirndler/wiringpi.net.sensors.git
pip3 install wiringpi
git clone https://github.com/depklyon/raspberrypi-python-tm1637.git
cd raspberrypi-python-tm1637
sudo python3 setup.py install
cd ..
git clone https://github.com/jdupl/dhtxx-rpi-python3.git
sudo cp dhtxx-rpi-python3/dhtxx.py /usr/local/lib/python3.5/dist-packages/
\end{console}


\lstset{language=bash, caption=/etc/motd.sh, label=LEDProgram, frame=single, basicstyle=\ttfamily
	\tiny, breakatwhitespace=false, showstringspaces=false, showtabs=false, tabsize=1 }
\lstinputlisting{motd.sh}

\begin{console}
echo "sh /etc/motd.sh" >> /home/pi/.profile
echo "sh /etc/motd.sh" >> /home/pi/.zprofile

reboot
\end{console}

Folgende Shell-Scripte k�nnen sp�ter Anf�ngern den Verbindungsaufbau und die Erstellung der ersten Beispielprogramms erleichtern. 

\begin{console} 
	cd ~
	mkdir scripts
	cd scripts
	wget --trust-server-names https://goo.gl/uwguQo 
	wget --trust-server-names https://goo.gl/rJwTLm
	chmod +x *.sh
\end{console}

Folgende Anweisungen sorgen daf�r, dass beim ersten Boot die root-Partition auf die maximale Kapazit�t der SD-Karte vergr��ert wird.   

\begin{console} 
	sudo wget -O /etc/init.d/resize2fs_once \
   https://raw.githubusercontent.com/smarkets/chef-raspberry-pi/master/files/default/resize2fs_once
	sudo chmod +x /etc/init.d/resize2fs_once
	sudo systemctl enable resize2fs_once

	sudo vi /boot/cmdline.txt 	
\end{console} 

Am Ende der ersten Zeile in der Datei cmdline.txt muss der Text\\ \textbf{init=/usr/lib/raspi-config/init\_resize.sh} \\
eingef�gt werden.

Folgende Anweisungen sorgen daf�r, dass beim ersten Boot ein neuer SSH-Key erzeugt wird.   

\begin{console} 
	sudo touch /boot/ssh
	sudo rm /etc/ssh/ssh_host_*
	cd /etc/systemd/system/multi-user.target.wants
	sudo ln -s /lib/systemd/system/regenerate_ssh_host_keys.service regenerate_ssh_host_keys.service
\end{console} 
%/etc/ssh/
%	moduli
%	ssh_config
%	sshd_config


Folgende Anweisungen setzen Protokollierdateien zur�ck.   

\begin{console} 
	sudo -i
	cd /var/log
	> auth.log; > daemon.log; > kern.log
	> messages; > wtmp; > debug
	> syslog; > lastlog; > faillog
  apt-get clean	
  history -c
  exit
  history -c
\end{console} 

% fill free space with zero
% dd if=/dev/zero of=dummy 
% rm dummy


%	
%/var/Log/  (alle Dateien 0 byte)
%	wtmp
%	lastlog
%	failllog
%	dkpg.log
%	btmp
%	bootstrap.log
%	alternatives.log
%	samba/
%	apt/
%		eipp.log.xz
%		history.log
%		term.log

\begin{console} 
	sudo halt
\end{console} 


%\begin{console} 
%	sudo fdisk -l /dev/mmcblk0
%\end{console}

%\begin{screensmall} 
%Units: sectors of 1 * 512 = 512 bytes
%Sector size (logical/physical): 512 bytes / 512 bytes
%I/O size (minimum/optimal): 512 bytes / 512 bytes
%Disklabel type: dos
%Disk identifier: 0xa8fe70f4
%
%Device         Boot Start     End Sectors  Size Id Type
%/dev/mmcblk0p1       8192   93802   85611 41,8M  c W95 FAT32 (LBA)
%/dev/mmcblk0p2      98304 3629055 3530752  1,7G 83 Linux
%\end{screensmall}

\section{Image erzeugen}

Hierzu muss die SD-Karte in einem anderen Linux System mit SD-Karten-Leseger�t eingesteckt werden.

\begin{console} 
	sudo umount /dev/mmcblk0p1 /dev/mmcblk0p2
	sudo dd if=/dev/mmcblk0 of=Raspjamming-<Jahr><Monat><Tag>.img bs=512
	7z a -t7z -mx5 Raspjamming-<Jahr><Monat><Tag>.img.7z Raspjamming-<Jahr><Monat><Tag>.img
\end{console} 




