
Nach der Einrichtung kann per SSH-Client eine Verbindungen zum Raspberry Pi hergestellt werden. Dazu muss in einem Terminal folgender Befehl eingegeben werden:

\begin{console} 
	ssh -X pi@raspberrypi.local
\end{console}

Um grafische Programme am Host anzeigen zu k�nnen, muss der Parameter \texttt{-X} angegeben werden. Dann wird eine X-Server Verbindung via SSH hergestellt. Verbindet man sich zum ersten Mal mit dem Raspberry Pi, so wird noch eine Sicherheitswarnung ausgegeben. Der kryptographische Schl�ssel f�r die Verbindung ist dem System noch nicht bekannt.%  und muss best�tigt werden.
\begin{screensmall}
The authenticity of host 'raspberrypi.local (192.168.137.10)' can't be established.
ECDSA key fingerprint is SHA256:Dcf3HYgE2GHnNnZ8Xhv8iJ9yA+zvfXBC9COm2eL9i0w.
Are you sure you want to continue connecting (yes/no)? 
Warning: Permanently added 'raspberrypi.local,192.168.137.10' (ECDSA) to the list of known hosts.
\end{screensmall}

Die Frage muss mit \texttt{yes} best�tigt werden. Anschlie�end muss das Default-Passwort von Raspbian \textbf{\texttt{raspberry}} eingegeben werden. Nun sollte man den Raspberry Pi Prompt \texttt{pi@rasbperrypi:\textasciitilde  \$} sehen. 

Wechselt man zu einem anderen Raspberry Pi mit den gleichen Namen, so kann es passieren, dass eine Fehlermeldung ausgegeben wird, weil sich der kryptographische Schl�ssel ge�ndert hat.

\begin{screensmall} 
@@@@@@@@@@@@@@@@@@@@@@@@@@@@@@@@@@@@@@@@@@@@@@@@@@@@@@@@@@@
@    WARNING: REMOTE HOST IDENTIFICATION HAS CHANGED!     @
@@@@@@@@@@@@@@@@@@@@@@@@@@@@@@@@@@@@@@@@@@@@@@@@@@@@@@@@@@@
\end{screensmall} 

%The ECDSA host key for raspberrypi.local has changed,
%and the key for the corresponding IP address 169.254.229.192
%is unknown. This could either mean that
%DNS SPOOFING is happening or the IP address for the host
%and its host key have changed at the same time.
%@@@@@@@@@@@@@@@@@@@@@@@@@@@@@@@@@@@@@@@@@@@@@@@@@@@@@@@@@@@
%@    WARNING: REMOTE HOST IDENTIFICATION HAS CHANGED!     @
%@@@@@@@@@@@@@@@@@@@@@@@@@@@@@@@@@@@@@@@@@@@@@@@@@@@@@@@@@@@
%IT IS POSSIBLE THAT SOMEONE IS DOING SOMETHING NASTY!
%Someone could be eavesdropping on you right now (man-in-the-middle attack)!
%It is also possible that a host key has just been changed.
%The fingerprint for the ECDSA key sent by the remote host is
%SHA256:Dcf2HYyE2GHnNpZ8Xhv8iJ9yj+zvfXBC9COm2eL9i0w.
%Please contact your system administrator.
%Add correct host key in /home/evil/.ssh/known_hosts to get rid of this message.
%Offending ECDSA key in /home/evil/.ssh/known_hosts:9
%remove with:
%ssh-keygen -f "/home/evil/.ssh/known_hosts" -R raspberrypi.local
%ECDSA host key for raspberrypi.local has changed and you have requested strict checking.


In diesem Fall muss man die Verbindung aus den bekannten Hosts l�schen. Hierf�r muss folgendes Kommando ausgef�hrt werden:

\begin{console} 
	ssh-keygen -R raspberrypi.local
\end{console}

Optional kann auch der Parameter \texttt{-o UserKnownHostsFile=/dev/null} beim ssh-Befehl hinzugef�gt werden. Dann erfolgt keine �berpr�fung der Verbindung. 

