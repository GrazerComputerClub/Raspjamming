%\subsection{Internet Zugriff} 
~\\
Nach der Einrichtung des Netzwerks, kann der Raspberry Pi Zero mit dem Namen "`raspberrypi.local"' erreicht werden. Um den Raspbery Pi Zero mit dem Internet verbinden zu k�nnen, m�ssen einige Einstellungen am Host %und Client
 gemacht werden. Man muss den Name des Netzwerkger�ts am Host-PC kennen, das mit dem Internet verbunden ist. Dies ermittelt man �ber die Netzwerkeinstellungen oder �ber den Terminal mit nmcli. Im Beispielfall ist der Name "`enp0s25"' das richtige Ger�t.

\begin{console} 
	nmcli d
\end{console} 

\begin{screensmall} 
	GER�T            TYP       STATUS           VERBINDUNG        
	enx000102030405  ethernet  verbunden        Raspberry Pi Zero 
	enp0s25          ethernet  verbunden        Netzwerkverbindung 1                
	lo               loopback  nicht verwaltet  --  
\end{screensmall}

Damit der Internetzugang f�r den Raspberry Pi Zero freigegeben wird, m�ssen am Host-PC folgende Befehle im Terminal eingeben werden. "`enp0s25"' muss durch den Namen des Netzwerkger�ts ersetzt werden, das mit dem Internet verbunden ist.

\begin{console} 
	sudo sysctl -w net.ipv4.ip_forward=1
	sudo iptables -t nat -A POSTROUTING -o enp0s25 -j MASQUERADE
\end{console}

