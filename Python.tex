\clearpage
\subsection{GPIOZero Python-Bibliothek / pinout Kommandozeilenprogramm}

GPIOZero ist eine der vielen Python Bibliotheken, welche genutzt werden kann
um auf die GPIOs der Raspberry Pi zuzugreifen. Sie wird unter der BSD Lizenz 
angeboten. Das Paket beinhaltet auch das Kommandozeilenprogramm "`pinout"'. Dieses zeigt 
Hardwareinformationen und die Pin-Anordnung in der Kommandozeile an.

\begin{console}
	pinout
\end{console} 

\begin{screensmall}
	.-------------------------.
	| oooooooooooooooooooo J8 |
	| 1ooooooooooooooooooo   |c
	---+       +---+ PiZero W|s
	 sd|       |SoC|   V1.1  |i
	---+|hdmi| +---+  usb pwr |
	`---|    |--------| |-| |-'
	
	Revision           : 9000c1
	SoC                : BCM2835
	RAM                : 512Mb
	Storage            : MicroSD
	USB ports          : 1 (excluding power)
	Ethernet ports     : 0
	Wi-fi              : True
	Bluetooth          : True
	Camera ports (CSI) : 1
	Display ports (DSI): 0
	
	J8:
	   3V3  (1) (2)  5V    
	 GPIO2  (3) (4)  5V    
	 GPIO3  (5) (6)  GND   
	 GPIO4  (7) (8)  GPIO14
	   GND  (9) (10) GPIO15
	GPIO17 (11) (12) GPIO18
	GPIO27 (13) (14) GND   
	GPIO22 (15) (16) GPIO23
	   3V3 (17) (18) GPIO24
	GPIO10 (19) (20) GND   
	 GPIO9 (21) (22) GPIO25
	GPIO11 (23) (24) GPIO8 
	   GND (25) (26) GPIO7 
	 GPIO0 (27) (28) GPIO1 
	 GPIO5 (29) (30) GND   
	 GPIO6 (31) (32) GPIO12
	GPIO13 (33) (34) GND   
	GPIO19 (35) (36) GPIO16
	GPIO26 (37) (38) GPIO20
	   GND (39) (40) GPIO21
	
	For further information, please refer to https://pinout.xyz/
\end{screensmall}

Die Bezeichnung der Pins ist sehr wichtig, da es unterschiedlichste Arten gibt die Pins anzusprechen. Die GPIOZero Bibliothek verwendet per Default die Broadcom Nummerierung (BCM numbering). Das hei�t, will man den physikalisch dritten Pin (direkt unter den 3.3V) schalten, muss man
wissen dass dieser die Bezeichnung GPIO2 besitzt und demzufolge die Pin-Nummer 2 verwendet werden muss. Technisch kann das Pin-Schema durch das Austauschen der Pin-Factories der GPIOZero Bibliothek erreicht werden (\emph{gpiozero.Device.pin\_factory}). Hier wird aber 
nicht n�her auf diese M�glichkeiten eingegangen.

