\subsection{GPIOZero Python-Bibliothek / pinout Tool}

GPIOZero ist eine der vielen Python Bibliotheken, welche genutzt werden kann
um auf die GPIOs der Raspberry PI zuzugreifen. Sie wird unter der BSD Lizenz 
angeboten.

Das Paket beinhaltet auch das Kommandozeilentool 'pinout'. Dieses zeigt 
Hardwareinformationen und die Pin-Anordnung in der Kommandozeile an.

\begin{screensmall}
.-------------------------.
| oooooooooooooooooooo J8 |
| 1ooooooooooooooooooo   |c
---+       +---+ PiZero W|s
 sd|       |SoC|   V1.1  |i
---+|hdmi| +---+  usb pwr |
`---|    |--------| |-| |-'

Revision           : 9000c1
SoC                : BCM2835
RAM                : 512Mb
Storage            : MicroSD
USB ports          : 1 (excluding power)
Ethernet ports     : 0
Wi-fi              : True
Bluetooth          : True
Camera ports (CSI) : 1
Display ports (DSI): 0

J8:
   3V3  (1) (2)  5V    
 GPIO2  (3) (4)  5V    
 GPIO3  (5) (6)  GND   
 GPIO4  (7) (8)  GPIO14
   GND  (9) (10) GPIO15
GPIO17 (11) (12) GPIO18
GPIO27 (13) (14) GND   
GPIO22 (15) (16) GPIO23
   3V3 (17) (18) GPIO24
GPIO10 (19) (20) GND   
 GPIO9 (21) (22) GPIO25
GPIO11 (23) (24) GPIO8 
   GND (25) (26) GPIO7 
 GPIO0 (27) (28) GPIO1 
 GPIO5 (29) (30) GND   
 GPIO6 (31) (32) GPIO12
GPIO13 (33) (34) GND   
GPIO19 (35) (36) GPIO16
GPIO26 (37) (38) GPIO20
   GND (39) (40) GPIO21

For further information, please refer to https://pinout.xyz/
\end{screensmall}

Die Bezeichnung der Pins ist sehr wichtig, da es zwei unterschiedliche Arten 
gibt die Pins anzusprechen. Die GPIOZero Bibliothek verwendet ausschlie�lich
die Broadcom Nummerierung (BCM numbering) und dies kann auch nicht auf die
physikalische Nummerierung (BOARD numbering) umgestellt werden. D.h. will
man den physikalisch dritten Pin (direkt unter den 3.3V) schalten, muss man
wissen dass dieser die Bezeichnung GPIO2 besitzt und demzufolge die Pin-Nummer 2
besitzt.
